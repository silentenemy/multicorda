\chapter{Модули}

Здесь будут кратко описаны все модули multicorda и порядок их загрузки.

\section{main.asm}

Этот файл описывает порядок ассемблирования модулей и общие настройки сборки.

\section{boot/}

В этой папке расположены файлы, отвечающие за первичную загрузку кода с диска и переключение в 32-битный защищённый режим.

\subsection{mbr.asm}

Этот файл ассемблируется в первые 512 байт дискеты и считается Master Boot Record - первичным загрузчиком для этой дискеты.

\subsection{stage2.asm}

Этот файл отвечает за настройку GDT, переключение в защищённый режим и ремаппинг PIC.

\section{acpi/}

В этой папке расположены файлы, отвечающие за работу с ACPI.

\subsection{rsdp\_lookup.asm}

Этот файл описывает процесс поиска Root System Description Pointer \cite{rsdp} - указателя на Root System Description Table \cite{rsdt}, основную таблицу описания системы ACPI.

\subsection{parse\_rsdt.asm}

Этот файл описывает процесс разбора RSDT, а именно поиска таблицы Multiple APIC Description Table \cite{madt} - таблицы описания множественных APIC-контроллеров \cite{apics}. Эта таблица используется для подсчёта процессорных ядер.

\subsection{madt\_count\_cpus.asm}

Этот файл описывает процесс разбора MADT. Для каждого процессорного ядра находится его собственный Local APIC, а также находится адрес I/O APIC.

\section{interrupts/}

В этой папке расположены файлы, отвечающие за настройку IDT \cite{idt} и прерываний.

\subsection{init\_interrupts.asm}

Этот файл отвечает за инициализацию IDT и настройку обработчиков прерываний.

\subsection{enable\_apics.asm}

Этот файл отвечает за инициализацию APIC-контроллеров.

\section{lib/}

В этой папке хранятся все функции, которые можно отнести к библиотечным.

\subsection{lib/acpi/}

В этой папке хранятся файлы, необходимые для разбора заголовков RSDP и таблиц ACPI.

\subsection{lib/interrupts/}

В этой папке хранятся файлы, необходимые для инициализации IDT, работы с (A)PIC, а также базовые обработчики прерываний.

\subsection{lib/keyboard/}

В этой папке хранится код драйвера клавиатуры.

\subsection{lib/biosvga*}

Эти файлы отвечают за работу с экраном через прерывания BIOS.

\subsection{lib/memvga*}

Эти файлы отвечают за работу с экраном через видеобуфер по адресу \verb|0B8000h|.

\subsection{lib/itohex.asm}

Этот файл содержит функции для перевода целых чисел в строковый шестнадцатиричный вид.

\subsection{lib/gdt.asm}

Этот файл содержит полную таблицу GDT.
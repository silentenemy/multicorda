\chapter{Основные особенности}

\section{Что такое multicorda}

multicorda - это миниатюрное ядро операционной системы, написанное на чистом ассемблере для архитектуры i386 и старше.

Так как это ядро уже хоть немного самостоятельной операционной системы, оно способно самостоятельно загружаться и выполнять минимальную настройку компьютера.

Кроме того, multicorda предоставляет некоторые ``библиотечные'' функции для пользовательских программ, работающих в её среде.

\section{Что может multicorda}
\begin{itemize}
\item Загружать себя с (виртуальной) дискеты;
\item Инициализировать GDT и IDT;
\item Переключать процессор в защищённый режим;
\item Перенастраивать программируемый контроллер прерываний i8259;
\item Работать с таблицами описания системы ACPI;
\item Извлекать информацию о количестве процессорных ядер;
\item Включать LAPIC;
\item Обрабатывать (некоторые) исключения процессора;
\item Получать прерывания от клавиатуры;
\item Работать с экраном.
\end{itemize}

\section{Ограничения}
\begin{itemize}
\item В настоящий момент multicorda не справляется с загрузкой с жесткого диска, только с дискеты;
\item Весь код выполняется в кольце привилегий 0;
\item В GDT есть только два перекрывающихся сегмента (кода и данных), которые покрывают 0-1 МиБ памяти;
\item В настоящий момент IOAPIC не инициализируется, поэтому работа с внешними источниками прерываний происходит через PIC 8259;
\item В настоящий момент драйвер клавиатуры может вызывать процессорные исключения.
\end{itemize}